\section{Introduction}
In an ever increasing cyber landscape just like in any increasing population, the level of crime increases with it. 
Resources enlisted to detect crime vary by type. A national entity doesn't \yanyan{Use "does not" instead of doesn't.} enlist local police to detect threats to national security.
Different types of criminal investigators look for different types of clues, ask different questions of different suspects, and will enlist different tiers of resources to help track them down. 
A national entity will enlist the CIA or FBI to hunt for national threats. \yanyan{I'm confused by the first few sentences.} 
The cyber "detective", should think in a similar fashion. 
That is; defining their domain, what exactly they are attempting to detect, the information that can be gathered from the susceptible environment, the methods which can be applied to that information in order to detect anomalies, and the resources needed to apply that method. 

At the highest level the goal of this paper is to assist the cyber detective; a security researcher, in being able to detect every type of cyber anomaly. \yanyan{Need a definition for cyber anomaly.} 
Of course \yanyan{"Of course" is informal.} just as crime slips through undetected so do cyber anomalies. 
In order to increase the success of detection, a formal classification of the approaches, and whats needed to carry them out, should be formally defined. 

To help address this issue this paper puts forth a taxonomy for detection literature. 
Using this taxonomy a researcher can organize their reading to ensure all relevant detection topics are covered. 
Additionally, after enough literature has been classified using this model gaps in research should become obvious. \yanyan{This paragraph reads hand-wavy. What gaps in research?}

The main taxonomy is discussed in depth in later sections and is shown in a logic diagram in figure 1. \yanyan{Figure~\ref{fig-model}. Please use ref, and Figure should be upper case.} 
In order to gratify readers with varying levels of interest in the depth of a detect literature taxonomy, other classification approaches are also presented. 
All classification approaches presented are congruent with one another and the final main taxonomy can be considered to have been bootstrapped from the others. 
All classification approaches can be applied by answering a series of questions about the literature. \yanyan{I got lost. Why you have to 
present a lot of classifications? Just to satisfy the readers?}

\subsection{Quick Classification}
On a quick browse a researcher can answer the following three simple questions regarding the paper being read. \yanyan{Which paper? A 
research paper in general, or this survey paper?}
\begin{itemize}
    \item What is being detected? \yanyan{In what regard? e.g., 
    is my smoke detector detecting smoke in my house?}
    \item What data is needed to perform that detection?
    \item What analysis methods are performed on the data to perform the detection?
\end{itemize}
Answering only these can provide a mechanism for which a researcher may be able to see where there is more or less saturation in the field.