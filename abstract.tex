\begin{abstract}
\yanyan{What detection literature? Do you mean anomaly detection?}
Detection literature can be categorized by their selection of a few key elements. The environment being monitored for anomalies, the type of anomaly being detected, the data needed to perform the detection, the methods used to perform analysis on that data, and specific tools to implement those methods. Additional surveys, taxonomies, and case studies can still fall under the proposed taxonomy. As far as the author knows there has been no single effort to classify detection literature using all of the above mentioned categories. This paper will provide a more complete classification for which a potential researcher, academic or professional, can use to classify detection literature. This classification can be used as an approach to determine if the researcher has read enough literature or implemented enough strategies to detect the anomalies relevant to their problem space. 
\yanyan{"has done their due diligence" sounds a bit too informal.}
\wayne{Tried to make it more formal}
\end{abstract}
\yanyan{Any keywords?}\wayne{Not sure what would make sense for keywords. Survey, taxonomy, cyber anomaly detection, are in the title, maybe the categories; environment, data collection, anomaly detection tools?}


\yanyan{When writing in latex, please keep each line to 70-80 characters.
When others open the file, they have to scroll all the way to the right
to see the entire paragraph. Also when you commit in git, the diff is
going to be a nightmare.}