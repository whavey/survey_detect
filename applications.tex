\section{Applications}
\subsection{In Research}
Future researchers could apply the taxonomy to their own literature, and that of literature they cite. 
This would allow them to produce a figure of a highlighted flow through the taxonomy.
Such a figure would allow a reader to quickly determine exactly what the authors are attempting to convey. 
Additionally, if the authors of future research believe their literature falls somehow outside the confines of this taxonomy, it may be interesting to show where exactly in the process they stand out.

Classification of enough literature using this taxonomy could reveal where research is lacking, or conversely over-saturated. 

\subsection{In Development and Testing}
This taxonomy could be applied as a testing model, or to define requirements, for a system monitoring platform. 
When developing such a platform the developers may consult a form of this taxonomy to see if they are creating a system that addresses all potential means of detecting cyber anomalies. 
Additionally, they may consult it to define tests for each detection type for the platform.