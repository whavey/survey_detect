\section{Detection Methods}
Detection methods are the real meat of cyber anomaly detection literature and so comprises the most critical and easily determined part of the taxonomy. 
Many of the categories presented here can and have been broken down much further in other surveys. 
It is recommended that for a deeper classification other surveys are explored in conjunction with what is presented below. 
With that being said, there will be breakdowns of the categories presented here representing the most commonly observed detection strategies in the literature analyzed for this survey.

The main classification model for detection methods is as follows. 
The highest level splits detect literature by whether the proposed method is a direct detection method or a support method. 
Direct detection is defined here as any method by which an anomaly detection is a direct result of the application of the method. 
A support method is any method that does not itself result in a detection but instead assists in performing detections in general.

Direct detections can be sub classified into three categories; machine learning techniques, analytics based techniques, and honey/canary techniques. 
Some additional sub classifications will be presented below but by no means is it a comprehensive list of attack types. 

Support methods can be sub classified into the following three categories; data organization, architecture, and query methods or languages. 
Examples for each subcategory will be given below however some additional explanation may be necessary for the support methods. 
Data organization is a support method for which detection literature may describe means for which store, retain, enrich, filter, and parse data. 
The cyber detective can see that these actions don't in themselves lead to a detection but instead allow other methods to perform more efficiently. 
Architecture is a support method classification for which the literature describes the system by which an analyst can perform their detect methods. 
This may include discussions on the high level components required for a detect system down to specific tools that could be included. 
Query methods or languages is a support method where the literature discusses ways of querying data which can include a new query language, or strategies such as querying data based on temporality. 
This is distinct from but integrated with running analytics.

\subsection{Other notes on Detection Methods}
While there is a clear distinction between the three main direct detection categories there can and often is overlap between them. 
It is plain to see how machine learning could be applied to analytic development. 
By allowing machine learning to inform what behavior is common in an environment and analytic can be developed to detect if there is uncommon behavior being performed.

Similarly, there can be overlap between the support methods described. 
An obvious example is that a query method or language proposed may require data to be stored or tagged in specific way thus encompassed both the query method and data organization sub categories.

An important sub classification that can be applied to all detection methods listed is whether or not the literature is discussing them in terms of automation or a manual process. 
In other words, do the authors present a means for which the detection or support method can be performed/setup through code?

A sub-categorization that can be made on detection literature is whether or not it discusses the human factor. 
Cyber anomaly detection is quickly becoming a widely studied subject and there are new methods for detection proposed all the time; however at the end of the day, the decision that something that was detected is actually an anomaly (especially a malicious one) is made by a human. 
Some literature will discuss how their methods actually performed in terms of the analysts that use it. 
The literature may discuss the users training level or compare their method between multiple types of users. 
In the same vein, the literature may go into specifics about the way alerts are shown to a user, or discuss the visualizations of the detections.

\subsection{Detection Method Sub Classification}
\begin{enumerate}
    \item Optionally apply to all: Automation vs Manual.
    \item Direct Detection
        \begin{itemize}
            \item Machine Learning\cite{buczak2016survey}\cite{dua2016data} 
            \begin{itemize}
                \item Training\\
                It has been observed that a large amount of detection literature focusing on machine learning techniques make use of the Darpa data set for training. 
                \item Algorithms
                \begin{itemize}
                    \item Two popular examples among the literature surveyed.
                    \begin{itemize}
                        \item Random Forest \cite{singh2014big}
                        \item K-means\cite{asif2011filtering}\cite{hajamydeen2016unsupervised}
                    \end{itemize}
                \end{itemize}
                \item clustering \cite{asif2011filtering}   
                \item classifying
                \item Regression analysis
            \end{itemize}
            \item Analytics\cite{cardenas2013big}
                \item Attack Graphs\cite{abraham2015predictive}
                \item Algorithms\cite{kim2013detection}
            \item Honey and Canary Technology \cite{jasek2013apt}\cite{saud2015towards}
        \end{itemize}
    \item Support Method
        \begin{itemize}
             \item Data Organization
                \begin{itemize}
                    \item Tagging
                \end{itemize}
                \item Architecture
                \item Query Languages\cite{mukherjee1994network}
                \\ Query languages present an interesting detection mechanism that is heavily related to analytics as they provide a means for them to be created them on the fly.
                Because of this, their effectivness is likely correlated with the skill and knowledge of the analyst.
                This may be an interesting research topic.
                \begin{itemize}
                    \item Temporality \cite{abraham2015predictive}\cite{abraham2014cyber}
                \end{itemize}
                \item Behavior definition
                    \begin{itemize}                                  \item White Lists\cite{yen2013beehive}
                    \end{itemize}
        \end{itemize}
\end{enumerate}